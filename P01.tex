\documentclass[10pt,openany]{book}
\usepackage[top=2cm,bottom=2cm,left=2cm,right=2cm,letterpaper]{geometry}
\usepackage{emptypage}
\usepackage[utf8]{inputenc}
\usepackage{steinmetz}
\usepackage{amssymb}
\usepackage{mathrsfs}
\usepackage{amsmath}
\usepackage{parskip}
\usepackage{scalerel}
\usepackage{sectsty}
\usepackage{graphicx}
\usepackage[spanish,es-noshorthands]{babel}
\usepackage[svgnames,dvipsnames,x11names,table]{xcolor}
\usepackage[listings,many]{tcolorbox,empheq}
\usepackage{colortbl}
\usepackage{booktabs}
\usepackage{lmodern}
\usepackage{utopia}
\usepackage[shortlabels]{enumitem}
\usepackage{tikz,tkz-euclide,pgf}
\usepackage{pgfplots}
\usepackage{array}
\usepackage[american voltages]{circuitikz}
\usepackage{float}
\usepackage{wrapfig}
\usepackage{varwidth}
\usepackage{latexsym}
\usepackage{caption}
\usepackage{subfigure}
\usepackage{multicol}
\usepackage[explicit]{titlesec}
\usepackage{titletoc}
\usepackage{etoolbox}
\usepackage{framed}
\usepackage{fancyhdr}
\usepackage{rotating}
\usepackage{pdfpages}
\usepackage{hyperref}

\pagestyle{plain}

\spanishdecimal{.}

\raggedbottom

\begin{document}

\begin{titlepage}
\thispagestyle{empty}
\noindent
\begin{minipage}[c]{0.25\textwidth}%Primera columna

\begin{figure}[H]
\includegraphics[width=0.7\linewidth]{Figuras/Escudo-UNAM.pdf}
\end{figure}
\end{minipage}\hfill\begin{minipage}[c]{0.75\textwidth}%Segunda columna
\begin{center}
\textbf{\Huge {UNIVERSIDAD NACIONAL\\[2mm]
AUTÓNOMA DE MÉXICO}}\vspace{5mm}
\hrule height 2.5pt \vspace{5mm}
\textbf{\Huge {FACULTAD DE INGENIERÍA}}
\end{center}
\end{minipage}

\noindent
\begin{minipage}[t]{0.25\textwidth}%Primera columna
%\begin{center}
\vspace{5mm}
\hspace{0.9cm}
\vrule width2.5pt height13cm
\hspace{8mm}
\vrule width2.5pt height13cm
%\end{center}
\end{minipage}\hfill\begin{minipage}[t]{0.75\textwidth}%Segunda columna

\begin{center}
\vspace{1cm}
\textbf{\huge {División de Ingeniería Eléctrica\\[2mm]
Departamento de Control y Robótica\\[2.5mm]
Lab. Circuitos Eléctricos}}\\[1.5cm]

\textbf{\huge {Práctica 1}}\\[6mm]
\textbf{\large {ANÁLISIS SINUSOIDAL PERMANENTE}}\\[6mm]
\textbf{\large {DE CIRCUITOS LINEALES}}\\[10mm]
\textbf{\large {Grupo: 11}}\\[10mm]
\textbf{\large {Profesor: Ing. Fernando Rivera}}\\[10mm]
\textbf{\large {Brigada: 4}}
\end{center}
\end{minipage}
\noindent
\begin{minipage}[T]{0.25\textwidth}%Primera columna
\begin{figure}[H]
\includegraphics[width=0.7\linewidth]{Figuras/Ingenieria.jpg}
\end{figure}
\end{minipage}\hfill\begin{minipage}[c]{0.75\textwidth}%Segunda columna
\vspace{2cm}
\textbf{Semestre 2025 - 2} \hfill \textbf{Fecha de realización: 21 de Enero de 2025}
\end{minipage}

\end{titlepage}

\tableofcontents

\newpage

\section*{Introducción}
\addcontentsline{toc}{section}{Introducción}
La teoría de circuitos eléctricos se basa en la representación y análisis de sistemas de primer y segundo orden. 

\subsection*{Sistemas de Primer Orden}
\addcontentsline{toc}{subsection}{Sistemas de Primer Orden}
Un sistema de primer orden tiene una función de transferencia de la forma:
\begin{equation}
    H(s) = \frac{M}{\tau s + 1}
\end{equation}
La respuesta al escalón se expresa como:
\begin{equation}
    yzs(t) = M k \left(1 - e^{-t/\tau} \right)
\end{equation}
La constante de tiempo $\tau$ es el tiempo que toma alcanzar el 63.2\% del valor final.

\subsection*{Sistemas de Segundo Orden}
\addcontentsline{toc}{subsection}{Sistemas de Segundo Orden}
Un sistema de segundo orden tiene una función de transferencia:
\begin{equation}
    H(s) = \frac{\omega_n^2}{s^2 + 2\zeta \omega_n s + \omega_n^2}
\end{equation}
Dependiendo del coeficiente de amortiguamiento $\zeta$, se tienen diferentes respuestas al escalón.

\section*{Desarrollo Experimental}
\addcontentsline{toc}{section}{Desarrollo Experimental}

\subsection*{Experimento 1: Resistencia Interna del Generador}
\addcontentsline{toc}{subsection}{Experimento 1: Resistencia Interna del Generador}

Medición de la resistencia interna del generador, $r_g$.
\begin{figure}[h]
    \centering
    \begin{circuitikz}
        \draw (3,0) to [short, *-] (6,0) 
            to[R, l_=$500\mathord{=}\Omega$] (6,3)
            to [short, -] (6,3)
            to [closing switch, l_=$S$, mirror, invert] (4,3)
            to [short, -*] (3,3)
            to [open, v^>=$v$] (3,0)
            to [short, -] (0,0)
            to [V, l=$v_g\mathord{=}E$] (0,3)
            to [short, -] (0,3)
            to [R, l=$r_g$] (2,3)
            to [short, -*] (3,3);
    \end{circuitikz}
    \caption{Circuito de ejemplo con una fuente, una resistencia y un capacitor.}
    \label{fig:circuito1}
\end{figure} \\
Construya el circuito eléctrico de la figura 1. La resistencia interna del generador, $r_g$ se puede determinar por medio de la ecuación
\begin{equation}
    \frac{\text{Amplitud de }v\text{ con }S\text{ cerrado}}{\text{Amplitud de }v\text{ con }S\text{ abierto}} = \frac{R}{r_g + R}
\end{equation}

\subsection*{Experimento 2: Inductancia de un Inductor}
\addcontentsline{toc}{subsection}{Experimento 2: Inductancia de un Inductor}

\begin{center}
    \begin{circuitikz}
        \draw (0,0) to[battery, v=V] (0,2)
              to[L=$L$, i>^=i] (2,2)
              to[R=$R$] (2,0)
              to[short] (0,0);
    \end{circuitikz}
\end{center}
La inductancia se calcula como:
\begin{equation}
    \tau = \frac{L}{R}
\end{equation}

\subsection*{Experimento 3: Capacitancia de un Capacitor}
\addcontentsline{toc}{subsection}{Experimento 3: Capacitancia de un Capacitor}

\begin{center}
    \begin{circuitikz}
        \draw (0,0) to[battery, v=V] (0,2)
              to[R=$R$] (2,2)
              to[C=$C$] (2,0)
              to[short] (0,0);
    \end{circuitikz}
\end{center}
La capacitancia se obtiene con:
\begin{equation}
    \tau = RC
\end{equation}

\subsection*{Experimento 4: Sistema de Segundo Orden}
\addcontentsline{toc}{subsection}{Experimento 4: Sistema de Segundo Orden}

\begin{center}
    \begin{circuitikz}
        \draw (0,0) to[battery, v=V] (0,2)
              to[L=$L$] (2,2)
              to[C=$C$] (4,2)
              to[R=$R$] (4,0)
              to[short] (0,0);
    \end{circuitikz}
\end{center}
Los parámetros del sistema se determinan con:
\begin{align}
    \omega_n &= \frac{1}{\sqrt{LC}} \\
    \zeta &= \frac{R}{2} \sqrt{\frac{C}{L}}
\end{align}

\section*{Conclusión}
\addcontentsline{toc}{section}{Conclusión}
A través de esta práctica, se verificaron las constantes de tiempo y parámetros de sistemas de primer y segundo orden. Los resultados experimentales fueron consistentes con la teoría.

\section*{Bibliografía}
\addcontentsline{toc}{section}{Bibliografía}
\begin{itemize}
    \item \url{https://www.electronics-tutorials.ws}
    \item \url{https://www.allaboutcircuits.com}
\end{itemize}
\end{document}